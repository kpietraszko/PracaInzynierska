\chapter*{Podsumowanie}
Celem pracy było stworzenie aplikacji usprawniającej płynność ruchu drogowego poprzez optymalizację sygnalizacji świetlnej. Udało się go osiągnąć, a aplikacja spełnia postawione wymagania funkcjonalne i pozafunkcjonalne. 
\section*{Symulacja ruchu drogowego}
Zaimplementowano symulację ruchu pojazdów reagujących na sygnalizatory i inne pojazdy, odpowiednio hamując lub przyspieszając w odpowiednich momentach. Ta symulacja stała się podstawą procesu uczenia decydując o ocenie każdego rozwiązania. Kod jest na tyle zoptymalizowany, że na współczesnych procesorach symulacja może bezproblemowo być uruchomiona ze 100-krotnym przyspieszeniem. Dalszą drogą do zwiększenie wydajności mogłoby być wprowadzenie wielowątkowości -- silnik Unity umożliwia wykorzystanie wszystkich wątków procesora, wymaga to jednak nieco innej struktury kodu i rozważenia przypadków charakterystycznych dla obliczeń równoległych. 
\section*{Algorytm ewolucyjny czy tam uczenie}