\chapter*{Podsumowanie}
Celem pracy było stworzenie aplikacji usprawniającej płynność ruchu drogowego poprzez optymalizację sygnalizacji świetlnej.
\section*{Symulacja ruchu drogowego}
Zaimplementowano symulację ruchu pojazdów reagujących na sygnalizatory i inne pojazdy, odpowiednio hamując lub przyspieszając w odpowiednich momentach. Ta symulacja stała się podstawą procesu optymalizacji decydując o ocenie każdego rozwiązania. Kod jest na tyle zoptymalizowany, że na współczesnych procesorach symulacja może bezproblemowo być uruchomiona ze 100-krotnym przyspieszeniem. Przy takim ustawieniu 100-pokoleniowy proces uczenia z 10 osobnikami na pokolenie trwa około godziny. Dalszą drogą do zwiększenia wydajności mogłoby być wprowadzenie wielowątkowości -- silnik Unity umożliwia wykorzystanie wszystkich wątków procesora, wymaga to jednak nieco innej struktury kodu i rozważenia przypadków charakterystycznych dla obliczeń równoległych.\\
Głównym wyzwaniem w implementacji symulacji było czasochłonne szukanie błędów w kodzie oraz jego optymalizacja.
\section*{Uczenie sygnalizacji świetlnej}
Zaimplementowano algorytm ewolucyjny, który optymalizuje parametry sygnalizatorów świetlnych. Udostępniono możliwość konfiguracji ustawień algorytmu. Na podstawie rozdziału \textit{Wyniki uczenia sygnalizacji} można stwierdzić, że algorytm jest dość skuteczny w znajdowaniu coraz lepszych rozwiązań. Jego implementacja była największym wyzwaniem w tej pracy, w szczególności dobór odpowiedniej techniki mutacji i krzyżowania osobników. Zastosowany algorytm można rozwinąć, wprowadzając dynamiczną regulację jego parametrów (jak choćby współczynnika mutacji) zależną od tempa postępów uczenia w poprzednich pokoleniach.
\paragraph{} 
Stabilność aplikacji jest na zadowalającym poziomie -- nawet przy wielogodzinnej pracy nie występują błędy.
\paragraph{} 
Postawiony cel pracy udało się osiągnąć, a aplikacja spełnia postawione wymagania funkcjonalne i pozafunkcjonalne. Można ją rozwinąć wprowadzając inne typy skrzyżowań lub dodając możliwość tworzenia własnych scenariuszy przez użytkownika.