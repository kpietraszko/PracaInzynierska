\chapter*{Użyte technologie}
Głównymi czynnikami decydującymi o doborze technologii były
\begin{itemize}
	\item Prostota interakcji z kartą graficzną,
	\item Dobre narzędzia do debugowania kodu,
	\item Jakość i kompletność dokumentacji,
	\item Wydajność,
	\item Możliwość prostej implementacji interfejsu graficznego,
	\item Wcześniejsze doświadczenie,
	\item Szybkość kompilacji -- istotna przy częstych zmianach w kodzie.
\end{itemize}
\section*{Unity}
Unity to wieloplatformowy silnik, którego głównym zastosowaniem jest tworzenie gier dwuwymiarowych i trójwymiarowych oraz symulacji. Wspiera wiele API graficznych (DirectX, OpenGL, Metal i Vulkan)\cite{UnityManualGraphicsApiSupport} pod wspólnym interfejsem. Dzięki temu twórcy mogą w łatwy sposób tworzyć wersje swoich aplikacji na wiele platform bez uciążliwego dostosowywania swojego kodu. Podobne uproszczenia dostępne są również w zakresie obsługi urządzeń wejścia (jak mysz, klawiatura, kontrolery, ekrany dotykowe), a także dźwięku. Na wspomnienie zasługuje również wbudowany edytor WYSIWYG\footnote{WYSIWYG (What You See Is What You Get) - ,,otrzymujesz to co widzisz''} interfejsu graficznego aplikacji. Silnik można też rozszerzać o własne narzędzia, na przykład graficzny edytor krzywych.