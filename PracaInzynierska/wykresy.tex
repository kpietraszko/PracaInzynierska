\chapter*{Wyniki uczenia}
W tym rozdziale przedstawione są wyniki uczenia sygnalizacji. 
\pgfplotstableread[col sep = comma]{p10,randOvertapOsc,3tourn10,pr1.csv}\evoA
\begin{figure}[h]
	\centering
	\begin{tikzpicture}
	\begin{axis}
	[
	grid=major,
	width=0.85\textwidth,
	xlabel=Pokolenie,
	ylabel=Czas najlepszego osobnika,
	xmin=0,
	xmax=100,
	ymin=0,
	ytick={0,10,...,160}
	]
	\addplot+[mark=none,thick] table {\evoA};
	\end{axis}
	\end{tikzpicture}
	\caption{}
	\label{fig:test2}
\end{figure}

\pgfplotstableread[col sep = comma]{pop10,tapOsc1over3,2tourn10of20.csv}\evoB
\begin{figure}[h]
	\centering
	\begin{tikzpicture}
	\begin{axis}
	[
	grid=major,
	width=0.85\textwidth,
	xlabel=Pokolenie,
	ylabel=Czas najlepszego osobnika,
	xmin=0,
	xmax=100,
	ymin=0,
	ytick={0,10,...,160}
	]
	\addplot+[mark=none,thick] table {\evoB};
	\end{axis}
	\end{tikzpicture}
	\caption{}
	\label{fig:test}
\end{figure}