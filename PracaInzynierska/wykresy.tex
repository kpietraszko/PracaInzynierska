\chapter*{Wyniki uczenia}
W tym rozdziale przedstawione są przykładowe wyniki uczenia sygnalizacji.\\
Rysunek~\ref{fig:evoA} przedstawia rezultaty z użyciem algorytmu opisanego w rozdziale \textit{Opis teoretyczny algorytmu ewolucyjnego}. W tym przykładzie rozmiar populacji pokolenia wynosił 10 osobników, a pozostałe ustawienia były domyślne. Jeśli porównać najlepszego osobnika  ostatniego i pierwszego pokolenia, to nastąpiła 26-procentowa poprawa czasu.
\pgfplotstableread[col sep = comma]{p10,randOvertapOsc,3tourn10,pr1.csv}\evoA
\begin{figure}[h]
	\centering
	\begin{tikzpicture}
	\begin{axis}
	[
	grid=major,
	width=1\textwidth,
	xlabel=Pokolenie,
	ylabel=Czas najlepszego osobnika {[s]},
	xmin=0,
	xmax=100,
	ymin=0,
	ytick={0,10,...,160}
	]
	\addplot+[mark=none,thick] table {\evoA};
	\end{axis}
	\end{tikzpicture}
	\caption{Wynik uczenia z 10 osobnikami w pokoleniu}
	\label{fig:evoA}
\end{figure}\\
\paragraph{}Rysunek~\ref{fig:evoB} przedstawia wyniki z wykorzystaniem tego samego algorytmu co wyżej, z tą różnicą, że rozmiar turnieju to 2 osobników, a najlepszy osobnik nie przechodził bezwarunkowo do kolejnego pokolenia. Rozmiar populacji pokolenia to ponownie 10 osobników, maksymalny współczynnik mutacji to 0,33, a do pozostałych ustawień użyto domyślnych wartości. Ostatecznie uzyskano tutaj 43-procentowy spadek czasu, jednak to w dużej mierze zasługa wysokiego czasu w pierwszym pokoleniu.
\pgfplotstableread[col sep = comma]{pop10,tapOsc1over3,2tourn10of20.csv}\evoB
\begin{figure}[h]
	\centering
	\begin{tikzpicture}
	\begin{axis}
	[
	grid=major,
	width=1\textwidth,
	xlabel=Pokolenie,
	ylabel=Czas najlepszego osobnika {[s]},
	xmin=0,
	xmax=100,
	ymin=0,
	ytick={0,10,...,200}
	]
	\addplot+[mark=none,thick] table {\evoB};
	\end{axis}
	\end{tikzpicture}
	\caption{Wynik uczenia inną wersją algorytmu}
	\label{fig:evoB}
\end{figure}
\paragraph{}
Z kolei na rysunku~\ref{fig:evoC} przedstawiono wyniki z wykorzystaniem stałego współczynnika mutacji równego 0,1 oraz populacji o rozmiarze 20 osobników. Uzyskano tu jedynie 6-procentowy spadek czasu między pierwszym a ostatnim pokoleniem.
\pgfplotstableread[col sep = comma]{pop20,newermut0.1.csv}\evoC
\begin{figure}[h]
	\centering
	\begin{tikzpicture}
	\begin{axis}
	[
	grid=major,
	width=1\textwidth,
	xlabel=Pokolenie,
	ylabel=Czas najlepszego osobnika {[s]},
	xmin=0,
	xmax=83,
	ymin=0,
	ytick={0,10,...,190}
	]
	\addplot+[mark=none,thick] table {\evoC};
	\end{axis}
	\end{tikzpicture}
	\caption{Wynik uczenia z 20 osobnikami w pokoleniu i stałym współczynnikiem mutacji}
	\label{fig:evoC}
\end{figure}