\chapter*{Wstęp}
%Wstęp od stworzenia swiata: komputery są pomocne również w zadaniach inżynieryjnych takich jak problemy optymalizacyjne np problem ustawienia optymalnych czasów (jak to ująć?)
\paragraph{}Komputery są obecnie niezwykle popularnymi urządzeniami znajdującymi zastosowanie w wielu dziedzinach życia. Są pomocne także w zadaniach inżynieryjnych takich jak problemy optymalizacyjne.
\section*{Problemy optymalizacyjne}
Problemy w wielu obszarach matematyki, inżynierii, ekonomii, medycyny i statystyki mogą być przedstawione w kategoriach optymalizacji.\\
Określenie problemu optymalizacyjnego rozpoczyna się od ustalenia zbioru zmiennych niezależnych lub parametrów. Często formułuje się również ograniczenia, które wyznaczają dozwolone wartości zmiennych. Inną istotną składową problemu optymalizacyjnego jest funkcja celu, której wartość zależy od zmiennych. Rozwiązaniem takiego problemu jest zbiór dozwolonych wartości zmiennych, dla których funkcja przyjmuje optymalną wartość (minimalną lub maksymalną, zależnie od badanego zagadnienia)\cite{9780122839528}.
\paragraph{} Do rozwiązywania problemów optymalizacyjnych często stosowane są metody sztucznej inteligencji. Są one szczególnie przydatne w złożonych problemach inżynieryjnych, gdzie tradycyjne techniki czasem zawodzą. 
\section*{Algorytmy ewolucyjne} Przykładowym typem takich metod są algorytmy ewolucyjne. Poszukują one optymalnego rozwiązania problemu w sposób, który czerpie inspirację z ewolucji gatunków. 
