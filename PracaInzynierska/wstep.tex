\chapter*{Wstęp}
%Wstęp od stworzenia swiata: komputery są pomocne również w zadaniach inżynieryjnych takich jak problemy optymalizacyjne np problem ustawienia optymalnych czasów (jak to ująć?)
\paragraph{}Komputery są obecnie niezwykle popularnymi urządzeniami znajdującymi zastosowanie w wielu dziedzinach życia. Są pomocne także w zadaniach inżynieryjnych takich jak problemy optymalizacyjne.
\section*{Problemy optymalizacyjne}
Problemy w wielu obszarach matematyki, inżynierii, ekonomii, medycyny i statystyki mogą być przedstawione w kategoriach optymalizacji.\\
Określenie problemu optymalizacyjnego rozpoczyna się od ustalenia zbioru zmiennych niezależnych lub parametrów. Często formułuje się również ograniczenia, które wyznaczają dozwolone wartości zmiennych. Inną istotną składową problemu optymalizacyjnego jest funkcja celu, której wartość zależy od zmiennych. Rozwiązaniem takiego problemu jest zbiór dozwolonych wartości zmiennych, dla których funkcja przyjmuje optymalną wartość (minimalną lub maksymalną, zależnie od badanego zagadnienia)\cite{9780122839528}.
\paragraph{} Do rozwiązywania problemów optymalizacyjnych często stosowane są metody sztucznej inteligencji. Dostarczają lepszych, szybszych i bardziej precyzyjnych rozwiązań niż konwencjonalne techniki, szczególnie w złożonych problemach inżynieryjnych. Spośród ich charakterystycznych cech warto wymienić następujące:
\begin{itemize}
	\item Metody te ,,pamiętają'' swoje poprzednie odkrycia,
	\item Dostosowują swoją wydajność (np. decydują o eksploracji lub eksploatacji), co przekłada się na unikanie utknięcia w minimum lokalnym,
	\item Mogą planować swoje działania długofalowo~\cite{Badar2013StudyOA}.
\end{itemize}
\section*{Algorytmy ewolucyjne} Przykładowym typem metod sztucznej inteligencji są algorytmy ewolucyjne. Poszukują one optymalnego rozwiązania problemu w sposób, który czerpie inspirację z mechanizmu ewolucji gatunków. Jedną z głównych cech wyróżniających ten zbiór technik jest to, że jednocześnie zmieniana jest pewna populacja rozwiązań, a jej przekształcenia przypominają procesy zachodzące w przyrodzie. Zrozumienie algorytmów ewolucyjnych wymaga przyswojenia pewnych kluczowych pojęć, takich jak dostosowanie, stanowiące ocenę danego rozwiązania, oraz krzyżowanie i mutacja, będące mechanizmami przekształcania populacji. \\Krzyżowanie tworzy nowe rozwiązanie łącząc cechy dwóch innych, podczas gdy mutacja w losowy sposób modyfikuje istniejące rozwiązanie. 
Przebieg algorytmu ewolucyjnego rozpoczyna się od utworzenia początkowej populacji rozwiązań, które zazwyczaj są generowane losowo. W związku z tym zakłada się, że jest to różnorodny zbiór, zawierający dobre i złe cechy.
Algorytm następnie generuje szereg kolejnych populacji (pokoleń), z pomocą wyżej wspomnianych mechanizmów przekształcających~\cite{Cohoon:2003:EAP:903758.903786}. Kluczowe jest tu zachowanie równowagi między wykorzystywaniem znanych pozytywnych cech (a więc krzyżowaniem) a odkrywaniem nowych możliwości w przestrzeni dopuszczalnych rozwiązań (a więc mutacją). Istnieją modele mające na celu zmniejszenie tego problemu. Przykładowo, działanie algorytmu można podzielić na fazy: pozyskiwania, akumulacji i wykorzystywania wiedzy~\cite{10.1371/journal.pone.0095693}. Innym rozwiązaniem jest przyjęcie zmiennego współczynnika mutacji danego pewną funkcją. Taką technikę przyjęto w tej pracy. 
\section*{Przykładowe zastosowania algorytmów ewolucyjnych} Do ciekawych zastosowań algorytmów ewolucyjnych należą m.in. antena statku kosmicznego~\cite{Lohn2006AutomatedAD}, czy też wentylator silnika~\cite{article}, których kształt został zaprojektowany przez taki algorytm. Obszary przejawiające duży potencjał to chociażby planowanie produkcji~\cite{Wall:1996:GAR:925320}, przewidywanie zmian temperatury na Ziemi\cite{Stanislawska:2012:MGT:2400749.2401077} i wiele innych~\cite{Steinbuch2010}.
\paragraph{}W tej pracy omówiono wykorzystanie algorytmu ewolucyjnego do usprawnienia płynności ruchu drogowego na skrzyżowaniu poprzez optymalizację sygnalizacji świetlnej. Stworzono aplikację przeprowadzającą proces uczenia z możliwością konfiguracji jego ustawień. Efektem końcowym pracy aplikacji jest wyświetlenie optymalnych parametrów sygnalizacji.
\section*{Usprawnienie sygnalizacji świetlnej jako problem optymalizacyjny}
Każda zmienna niezależna to czas, przez jaki wybrane sygnalizatory są zielone. Ewaluacja funkcji celu każdorazowo wymaga przeprowadzenia symulacji ruchu pojazdów, którą w tym celu zaimplementowano. Symulacja trwa, dopóki określona liczba samochodów nie opuści skrzyżowania, a czas jej trwania decyduje o ocenie danego osobnika. \\
\section*{Zawartość pracy}
W rozdziale \textit{Użyte technologie} opisano wykorzystane narzędzia i język programowania, a także czynniki decydujące o wyborze silnika.\\
W rozdziale \textit{Projekt aplikacji} opisano wymagania funkcjonalne i pozafunkcjonalne projektu oraz główny przypadek użycia.\\
Rozdział \textit{Opis teoretyczny algorytmu ewolucyjnego} szczegółowo przedstawia algorytm ewolucyjny, który zaimplementowano w aplikacji.\\
W rozdziale \textit{Opis aplikacji} zaprezentowano aplikację z punktu widzenia użytkownika i wyjaśniono jej funkcjonowanie. Opisano też ustawienia, które może zmieniać użytkownik, a także dodatkowe narzędzia powstałe w trakcie implementacji.\\
W rozdziale \textit{Wyniki uczenia sygnalizacji} opisano rezultaty kilku przykładowych przebiegów procesu uczenia.\\
W podsumowaniu opisano, czego udało się dokonać w ramach pracy. Przedstawiono też największe wyzwania i możliwe kierunki rozwoju aplikacji.